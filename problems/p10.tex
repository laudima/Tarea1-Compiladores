\begin{Pro}
Del ejercicio anterior, utiliza las regex de las expresiones booleanas para definir el AF, conviértelo en AFD y redúcelo. Pruébalo con las siguientes expresiones para obtener los tokens:

\begin{enumerate}
    \item[(a)] $p \land q$
    \item[(b)] $\textbf{True} \land \lnot(p \lor q)$
    \item[(c)] $\lnot(x \land y) \lor (p \land q)$
\end{enumerate}

\end{Pro}


Para convertirlo en AF usando Thompson creamos un estado inicial y un estado final, y vamos añadiendo los estados y transiciones según las regex que definimos en el ejercicio anterior. Con lo que obtenemos:

\begin{figure}[H]
    \centering
    \includegraphics[width=0.8\textwidth]{images/ejercicio10-1.jpg}
    \caption{AF para lexer expresiones booleanas}
    \label{fig:my_label}
\end{figure}


Primero vamos a obtener la epsilon-cerradura de cada uno de los estados:

\begin{itemize}
    \item $\epsilon-closure(\{q_0\}) = \{q_0, q_1, q_2, q_3, q_4\}$
    \item $\epsilon-closure(\{q_1\}) = \{q_1\}$
    \item $\epsilon-closure(\{q_2\}) = \{q_2\}$
    \item $\epsilon-closure(\{q_3\}) = \{q_3\}$
    \item $\epsilon-closure(\{q_4\}) = \{q_4\}$
    \item $\epsilon-closure(\{q_5\}) = \{q_5\}$
    \item $\epsilon-closure(\{q_6\}) = \{q_6\}$
    \item $\epsilon-closure(\{q_7\}) = \{q_7\}$
    \item $\epsilon-closure(\{q_8\}) = \{q_8\}$
    \item $\epsilon-closure(\{q_9\}) = \{q_9\}$
    \item $\epsilon-closure(\{q_{10}\}) = \{q_{10}\}$
    \item $\epsilon-closure(\{q_{11}\}) = \{q_{11}\}$
    \item $\epsilon-closure(\{q_{12}\}) = \{q_{12}\}$
    \item $\epsilon-closure(\{q_{13}\}) = \{q_{13}\}$
    \item $\epsilon-closure(\{q_{14}\}) = \{q_{14}\}$
    \item $\epsilon-closure(\{q_{15}\}) = \{q_{15}\}$
\end{itemize}

Ahora vamos a obtener las transiciones de cada uno de los estados:

\begin{table}[h!]
\centering
\begin{tabular}{|c|c|c|c|c|c|c|c|c|c|c|c|c|}
\hline
Estado &T & r & u & e & F & a & l &s &[a-z]-[r,u,e,a,l,s] &$\land/\lor/\lnot$ & (, )\\ \hline
\hline
$\{q_0, q_1, q_2, q_3, q_4\}$ & $\{q_9\}$ & $\{q_6\}$ & $\{q_6\}$ & $\{q_{6}\}$ & $\{q_{12}\}$ & $\{q_6\}$ & $\{q_6\}$ & $\{q_6\}$ & $\{q_6\}$ & $\{q_7\}$& $\{q_8\}$\\ \hline
\end{tabular}
\end{table}

En este punto tenemos como nuevos estados $\{q_6\}, \{q_7\}, \{q_8\}, \{q_9\}, \{q_{12}\}$. Vamos a agregarlos a la tabla de transiciones:

\begin{table}[h!]
\centering
\begin{tabular}{|c|c|c|c|c|c|c|c|c|c|c|c|c|}
\hline
Estado &T & r & u & e & F & a & l &s &[a-z]-[r,u,e,a,l,s] &$\land/\lor/\lnot$ & (, )\\ \hline
\hline
$\{q_0, q_1, q_2, q_3, q_4\}$ & $\{q_9\}$ & $\{q_6\}$ & $\{q_6\}$ & $\{q_{6}\}$ & $\{q_{12}\}$ & $\{q_6\}$ & $\{q_6\}$ & $\{q_6\}$ & $\{q_6\}$ & $\{q_7\}$& $\{q_8\}$\\ \hline
$\{q_6\}$ & $\emptyset$ & $\emptyset$ & $\emptyset$ & $\emptyset$ & $\emptyset$ & $\emptyset$ & $\emptyset$ & $\emptyset$ & $\emptyset$ & $\emptyset$ & $\emptyset$\\ \hline
$\{q_7\}$ & $\emptyset$ & $\emptyset$ & $\emptyset$ & $\emptyset$ & $\emptyset$ & $\emptyset$ & $\emptyset$ & $\emptyset$ & $\emptyset$ & $\emptyset$ & $\emptyset$\\ \hline
$\{q_8\}$ & $\emptyset$ & $\emptyset$ & $\emptyset$ & $\emptyset$ & $\emptyset$ & $\emptyset$ & $\emptyset$ & $\emptyset$ & $\emptyset$ & $\emptyset$ & $\emptyset$\\ \hline
$\{q_9\}$ & $\emptyset$ & $\{q_{10}\}$ & $\emptyset$ & $\emptyset$ & $\emptyset$  & $\emptyset$ & $\emptyset$ & $\emptyset$ & $\emptyset$ & $\emptyset$ & $\emptyset$\\ \hline
$\{q_{12}\}$ & $\emptyset$ & $\emptyset$ & $\emptyset$ & $\emptyset$ & $\emptyset$  & $\{q_{13}\}$ & $\emptyset$ & $\emptyset$& $\emptyset$ & $\emptyset$ & $\emptyset$\\ \hline
\end{tabular}
\end{table}

Continuamos con las nuevas transiciones que hemos obtenido y obtenemos la tabla completa:

\begin{table}[H]
\centering
\begin{tabular}{|c|c|c|c|c|c|c|c|c|c|c|c|c|}
\hline
Estado &T & r & u & e & F & a & l &s &[a-z]-[r,u,e,a,l,s] &$\land/\lor/\lnot$ & (, )\\ \hline
\hline
$\{q_0, q_1, q_2, q_3, q_4\}$ & $\{q_9\}$ & $\{q_6\}$ & $\{q_6\}$ & $\{q_{6}\}$ & $\{q_{12}\}$ & $\{q_6\}$ & $\{q_6\}$ & $\{q_6\}$ & $\{q_6\}$ & $\{q_7\}$& $\{q_8\}$\\ \hline
$\{q_6\}$ & $\emptyset$ & $\emptyset$ & $\emptyset$ & $\emptyset$ & $\emptyset$ & $\emptyset$ & $\emptyset$ & $\emptyset$ & $\emptyset$ & $\emptyset$ & $\emptyset$\\ \hline
$\{q_7\}$ & $\emptyset$ & $\emptyset$ & $\emptyset$ & $\emptyset$ & $\emptyset$ & $\emptyset$ & $\emptyset$ & $\emptyset$ & $\emptyset$ & $\emptyset$ & $\emptyset$\\ \hline
$\{q_8\}$ & $\emptyset$ & $\emptyset$ & $\emptyset$ & $\emptyset$ & $\emptyset$ & $\emptyset$ & $\emptyset$ & $\emptyset$ & $\emptyset$ & $\emptyset$ & $\emptyset$\\ \hline
$\{q_9\}$ & $\emptyset$ & $\{q_{10}\}$ & $\emptyset$ & $\emptyset$ & $\emptyset$  & $\emptyset$ & $\emptyset$ & $\emptyset$ & $\emptyset$ & $\emptyset$ & $\emptyset$\\ \hline
$\{q_{12}\}$ & $\emptyset$ & $\emptyset$ & $\emptyset$ & $\emptyset$ & $\emptyset$  & $\{q_{13}\}$ & $\emptyset$ & $\emptyset$& $\emptyset$ & $\emptyset$ & $\emptyset$\\ \hline
$\{q_{10}\}$ & $\emptyset$ &  $\emptyset$&  $\{q_{11}\}$& $\emptyset$ & $\emptyset$  & $\emptyset$ & $\emptyset$ & $\emptyset$ & $\emptyset$ & $\emptyset$ & $\emptyset$\\ \hline
$\{q_{11}\}$ & $\emptyset$ &  $\emptyset$& $\emptyset$ &  $\{q_{5}\}$& $\emptyset$  & $\emptyset$ & $\emptyset$ & $\emptyset$ & $\emptyset$ & $\emptyset$ & $\emptyset$\\ \hline
$\{q_{13}\}$ & $\emptyset$ &  $\emptyset$& $\emptyset$ & $\emptyset$ & $\emptyset$  & $\emptyset$ & $\{q_{14}\}$ & $\emptyset$ & $\emptyset$ & $\emptyset$ & $\emptyset$\\ \hline
$\{q_{14}\}$ & $\emptyset$ &  $\emptyset$& $\emptyset$ & $\emptyset$ & $\emptyset$  & $\emptyset$ & $\emptyset$ & $\{q_{15}\}$ & $\emptyset$ & $\emptyset$ & $\emptyset$\\ \hline
$\{q_{15}\}$ & $\emptyset$ &  $\emptyset$& $\emptyset$ & $\{q_{5}\}$ & $\emptyset$  & $\emptyset$ & $\emptyset$ & $\emptyset$ & $\emptyset$ & $\emptyset$ & $\emptyset$\\ \hline
$\{q_{5}\}$ & $\emptyset$ &  $\emptyset$& $\emptyset$ & $\emptyset$ & $\emptyset$  & $\emptyset$ & $\emptyset$ & $\emptyset$ & $\emptyset$ & $\emptyset$ & $\emptyset$\\ \hline
\end{tabular}

\end{table}


Llamaremos $A=\{q_0, q_1, q_2, q_3, q_4\}$ los demas estados al quedar unitarios conservaran su nombre. Tenemos entonces el siguiente AFD:


Note que la transición de $A$ al estado $\{q_6\}$ se da con las letras $r, u, a, l, s$ y cualquier letra de la $a$ a la $z$ que no sea $r, u, e, a, l, s$. Por lo que podemos agruparlas en una sola transición con la etiqueta $[a-z]$. Habiendo aclarado esto, tenemos el siguiente AFD:


\begin{figure}[h!]
    \centering
    \includegraphics[width=0.8\textwidth]{images/ejercicio10-2.jpg}
    \caption{AFD para lexer expresiones booleanas}
    \label{fig:my_label}
\end{figure}


