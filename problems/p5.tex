\begin{Pro}
    Realiza una breve investigación acerca del esquema de firma de Merkle(MSS). La investigación debe 
    responder como mínimo las siguientes preguntas:
    \begin{itemize}
        \item ¿Qué es un árbol de Merkle?
        \item ¿De qué manera se generan firmas? 
        \item ¿De qué manera se verifican las firmas? 
    \end{itemize}
\end{Pro}

\begin{proof}
    \hspace{5mm}
    \begin{itemize}
        \item \textbf{Árbol de Merkle:} Se puede considar como una estructura de datos 
        en donde cada hoja es un hash de un bloque de datos y cada nodo interno es un hash de las concatenaciones
        de los nodos hijos. La raíz del árbol es el hash de la concatenación de las hojas. Nacieron en 1980 con Ralph Merkle.

        \item \textbf{Generación de firmas:} Hacemos un hash sobre todas las hojas del árbol( que pueden ser las palabras que queremos cifrar). 
        \item \textbf{Verificación de firmas:} Se puede verificar la firma de manera eficiente utilizando el hash de la raíz del árbol y el hash de las hojas.
        
        Aqui esta un ejemplo del algoritmo de firma de Merkle:
        
        \lstinputlisting{code/arbol_merkle.py}
        
    \end{itemize}
\end{proof}
