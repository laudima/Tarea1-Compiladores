\begin{Pro}
    Para las siguientes expresiones regulares, da el lenguaje que definen: 
    \begin{enumerate}
        \item $[ab][cd\epsilon]$ 
        \item $[a-zA-Z]^*at^*$
        \item $ca[tr]$
    \end{enumerate}
\end{Pro}
\begin{proof}
    \hspace{1cm}
    \begin{enumerate}
        \item $[ab][cd\epsilon]$ define el lenguaje $\{a,b,ac,ad,bc, bd\}$
        \begin{align*}
            L([ab][cd\epsilon]) &= L([ab]) \cdot L([cd\epsilon]) \\
            &=  ( L(a) \cup L(b) )\cdot (L(c) \cup L(d) \cup L(\epsilon)) \\
            &= \{a,b\} \cdot \{c,d,\epsilon\} \\
            &= \{a,b,ac,ad,bc,bd \}
        \end{align*}
        \item $[a-zA-Z]^*at^*$ define el lenguaje de (letras)a$t^*$, es decir,todas cadenas de letras seguidas de \texttt{a} y luego de cero o más \texttt{t}. Por ejemplo: \texttt{at}, \texttt{bat}, \texttt{cat}, \texttt{a}, \texttt{rattt}, etc.
        \begin{align*}
            L([a-zA-Z]^*at^*) &= L([a-zA-Z]^*) \cdot L(a) \cdot L(t^*) \\
            &= (L(a) \cup L(b) \cup ... \cup L(z) \cup L(A) \cup ... \cup L(Z))^* \cdot L(a) \cdot (L(t))^* \\
            &= \{  a, b, ..., z, A, ..., Z\}^* \cdot \{a\} \cdot  \{t\}^* \\
        \end{align*}
        \item $ca[tr]$ define el lenguaje $\{cat, car\}$
        \begin{align*}
            L(ca[tr]) &= L(c) \cdot L(a) \cdot L([tr]) \\
            &= L(c) \cdot L(a) \cdot (L(t) \cup L(r)) \\
            &= \{c\} \cdot \{a\} \cdot \{t, r\} \\
            &= \{cat, car\}
        \end{align*}
    \end{enumerate}
\end{proof}
\newpage