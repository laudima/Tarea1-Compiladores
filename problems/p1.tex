\begin{Pro}
Indica los valores asignados a $w$, $x$, $y$ y $z$. en los siguientes dos códigos estructurados por bloques.
Muestrala tabla de símbolos en cada bloque con una implementación imperativa en cada caso:
\end{Pro}

\begin{multicols}{2}
\begin{lstlisting}[language=C++]
int w,x,y,z;
int i = 4; int j = 5;
{
    int j =7;
    i =6; 
    w = i+j;
}
x = i+j;
{
    int i = 8;
    y = i+j;
}
z = i+j;
\end{lstlisting}
\begin{lstlisting}[language=C++]
int w,x,y,z;
int i = 3; int j = 4;
{
    int i = 5;
    w = i+j;
}
x = i+j;
{
    int j = 6;
    i = 7;
    y = i+j;
}
z = i+j;
\end{lstlisting}

\end{multicols}
\begin{proof}
    \hspace{5mm}

\begin{table}[h!]
\begin{tabular}{lll}
Variable                    & Scope    & Comentario                                                 \\
int i =4                    & Bloque 1 & Nueva variable                                             \\
int j = 5                   & Bloque 1 & Nueva variable                                             \\
int j = 7                   & Bloque 2 & Nueva variable, solo queda en el Bloque 2                  \\ 
i=6                         & Bloque 2 & Afecta al Bloque 1                                         \\
x = 9                       & Bloque 1 & Afecta al Bloque 1                                         \\
int i =8                    & Bloque 3 & Nueva variable, solo queda en el Bloque 3                  \\
y = 13                      & Bloque 3 & Afecta al Bloque 1                                         \\
z = 6 + 5                   & Bloque 1 & El valor de i cambio, pero el de j es el del primer reglon \\
Final                       &          & w = 13, x = 9, y = 13, z = 11                             
\end{tabular}
\end{table}
\end{proof}
