\begin{Pro}
    Define una función recursiva que compute los prefijos de una expresión regular. La
    base de tal función recursiva es: 
    \begin{align*}
        prefix(\varepsilon) = \{\varepsilon\}\\
        prefix(a) = \{a\}\\ 
    \end{align*}
    Completa la definición. 
\end{Pro}

\begin{proof}
    \hspace{5mm}
    Para definir los prefijos de una expresión regular, podemos considerar las siguientes reglas:

    Podemos definir las reglas que aplicamos para cada una de las operaciones básicas de las expresiones regulares, 
    si tenemos que $R$ y $S$ son expresiones regulares y $a$ es un símbolo, entonces:
    \begin{align*}
        prefix(\epsilon) & = \{\epsilon\}\\
        prefix(a) & = \{\epsilon, a\}\\
        prefix(RS) & = prefix(R) \cup R \circ prefix(S)\\
        prefix(R|S) & = prefix(R) \cup prefix(S)\\
        prefix(R*) & = R^* \circ prefix(R)
    \end{align*}

\end{proof}