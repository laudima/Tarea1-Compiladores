\begin{Pro}
Divide el siguiente programa en C++ en lexemas y genera los tokens correspondientes:
\end{Pro}
\begin{lstlisting}[language=C++]
    float limitedSquare(x) float x; {
        /* return x-squared, bit never mover than 100 */
        return(x <= -10.0 || x>=10.0) ? 100 : x*x;
    }
\end{lstlisting}

\begin{proof}
    \hspace{5mm}

Después del escaneo no tenemos comentarios. Ni espacios en blanco. 
Un token es una pareja $$<tipo, valor>$$

Tenemos los siguientes tipos de tokens:
\begin{itemize}
    \item Keywords (Kw) como \texttt{return, if, else, while, ...}
    \item Operadores (Op) como \texttt{+, -, *, /, =, ==, <=, >=, ...}
    \item Identificadores (Id) como \texttt{variable1, funcion2, ...}
    \item Constantes (Const) como \texttt{3, 4.5, 'a', "hola", ...}
    \item Símbolos de puntuación (Punct) como \texttt{(, ), {, }, ;, ...}
\end{itemize}

En este caso tenemos los siguientes tokens:
\begin{lstlisting}
<Kw, float> 
<Id, limitedSquare>
<Punct, (>
<Id, x>
<Punct, )>
<Kw, float>
<Id, x>
<Punct, ;>
<Punct, {>
<Kw, return>
<Punct, (>
<Id, x>
<Op, <=>
<Const, -10.0>
<Op, ||>
<Id, x>
<Op, >=>
<Const, 10.0>
<Punct, )>
<Op, ?>
<Const, 100>
<Op, :>
<Id, x>
<Op, *>
<Id, x>
<Punct, ;>
<Punct, }>
\end{lstlisting}
\end{proof}
