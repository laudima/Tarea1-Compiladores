\begin{Pro}
 Define un lexer para las expresiones booleanas que considere los siguientes puntos:


\begin{enumerate}
    \item[(a)] Que considere las constantes \textbf{True} y \textbf{False} como tokens \textbf{const}.
    
    \item[(b)] Que considere las variables (\textbf{var}) $x, y, z, p, q, r$, etc.

    \item[(c)] Que considere los operadores binarios (\textbf{binop}) básicos: $\land, \lor$ y $\lnot$.

    \item[(d)] Define las regex de este para este lexer.
\end{enumerate}

\end{Pro}

El trabajo del lexer solo es identificar los tokens en la cadena de entrada.

Primero vamos a definir las regex para cada uno de los tokens que necesitamos:
\begin{itemize}
    \item Para constantes \textbf{True} y \textbf{False}:
    \begin{align*}
        const = [True | False]
    \end{align*}

    \item Para variables (\textbf{var}) $x, y, z, p, q, r$ solo consideramos de longitud 1 como suelen ser las variables en lógica proposicional:
    \begin{align*}
        var = [a-z]
    \end{align*}

    \item Para los operadores (\textbf{binop}) básicos: $\land, \lor$:
    \begin{align*}
        binop = [\land | \lor | \lnot]
    \end{align*}
 

    \item Para los signos de puntuación (\textbf{punct}): paréntesis de apertura y cierre:
    \begin{align*}
        punct = [ ( | ) ]
    \end{align*}

\end{itemize}


